% Options for packages loaded elsewhere
\PassOptionsToPackage{unicode}{hyperref}
\PassOptionsToPackage{hyphens}{url}
%
\documentclass[
]{book}
\title{Mitos y verdades de las Elecciones Subnacionales en Perú}
\author{Área de investigación e incidencia de la Escuela de Gobierno y Políticas Públicas - PUCP}
\date{Equipo de trabajo: Jorge Aragón, Marylía Cruz, Karina Alcantara, Paolo Sanchez, Valeria Pinchi y Cinthya Villanueva}

\usepackage{amsmath,amssymb}
\usepackage{lmodern}
\usepackage{iftex}
\ifPDFTeX
  \usepackage[T1]{fontenc}
  \usepackage[utf8]{inputenc}
  \usepackage{textcomp} % provide euro and other symbols
\else % if luatex or xetex
  \usepackage{unicode-math}
  \defaultfontfeatures{Scale=MatchLowercase}
  \defaultfontfeatures[\rmfamily]{Ligatures=TeX,Scale=1}
\fi
% Use upquote if available, for straight quotes in verbatim environments
\IfFileExists{upquote.sty}{\usepackage{upquote}}{}
\IfFileExists{microtype.sty}{% use microtype if available
  \usepackage[]{microtype}
  \UseMicrotypeSet[protrusion]{basicmath} % disable protrusion for tt fonts
}{}
\makeatletter
\@ifundefined{KOMAClassName}{% if non-KOMA class
  \IfFileExists{parskip.sty}{%
    \usepackage{parskip}
  }{% else
    \setlength{\parindent}{0pt}
    \setlength{\parskip}{6pt plus 2pt minus 1pt}}
}{% if KOMA class
  \KOMAoptions{parskip=half}}
\makeatother
\usepackage{xcolor}
\IfFileExists{xurl.sty}{\usepackage{xurl}}{} % add URL line breaks if available
\IfFileExists{bookmark.sty}{\usepackage{bookmark}}{\usepackage{hyperref}}
\hypersetup{
  pdftitle={Mitos y verdades de las Elecciones Subnacionales en Perú},
  pdfauthor={Área de investigación e incidencia de la Escuela de Gobierno y Políticas Públicas - PUCP},
  hidelinks,
  pdfcreator={LaTeX via pandoc}}
\urlstyle{same} % disable monospaced font for URLs
\usepackage{color}
\usepackage{fancyvrb}
\newcommand{\VerbBar}{|}
\newcommand{\VERB}{\Verb[commandchars=\\\{\}]}
\DefineVerbatimEnvironment{Highlighting}{Verbatim}{commandchars=\\\{\}}
% Add ',fontsize=\small' for more characters per line
\usepackage{framed}
\definecolor{shadecolor}{RGB}{248,248,248}
\newenvironment{Shaded}{\begin{snugshade}}{\end{snugshade}}
\newcommand{\AlertTok}[1]{\textcolor[rgb]{0.94,0.16,0.16}{#1}}
\newcommand{\AnnotationTok}[1]{\textcolor[rgb]{0.56,0.35,0.01}{\textbf{\textit{#1}}}}
\newcommand{\AttributeTok}[1]{\textcolor[rgb]{0.77,0.63,0.00}{#1}}
\newcommand{\BaseNTok}[1]{\textcolor[rgb]{0.00,0.00,0.81}{#1}}
\newcommand{\BuiltInTok}[1]{#1}
\newcommand{\CharTok}[1]{\textcolor[rgb]{0.31,0.60,0.02}{#1}}
\newcommand{\CommentTok}[1]{\textcolor[rgb]{0.56,0.35,0.01}{\textit{#1}}}
\newcommand{\CommentVarTok}[1]{\textcolor[rgb]{0.56,0.35,0.01}{\textbf{\textit{#1}}}}
\newcommand{\ConstantTok}[1]{\textcolor[rgb]{0.00,0.00,0.00}{#1}}
\newcommand{\ControlFlowTok}[1]{\textcolor[rgb]{0.13,0.29,0.53}{\textbf{#1}}}
\newcommand{\DataTypeTok}[1]{\textcolor[rgb]{0.13,0.29,0.53}{#1}}
\newcommand{\DecValTok}[1]{\textcolor[rgb]{0.00,0.00,0.81}{#1}}
\newcommand{\DocumentationTok}[1]{\textcolor[rgb]{0.56,0.35,0.01}{\textbf{\textit{#1}}}}
\newcommand{\ErrorTok}[1]{\textcolor[rgb]{0.64,0.00,0.00}{\textbf{#1}}}
\newcommand{\ExtensionTok}[1]{#1}
\newcommand{\FloatTok}[1]{\textcolor[rgb]{0.00,0.00,0.81}{#1}}
\newcommand{\FunctionTok}[1]{\textcolor[rgb]{0.00,0.00,0.00}{#1}}
\newcommand{\ImportTok}[1]{#1}
\newcommand{\InformationTok}[1]{\textcolor[rgb]{0.56,0.35,0.01}{\textbf{\textit{#1}}}}
\newcommand{\KeywordTok}[1]{\textcolor[rgb]{0.13,0.29,0.53}{\textbf{#1}}}
\newcommand{\NormalTok}[1]{#1}
\newcommand{\OperatorTok}[1]{\textcolor[rgb]{0.81,0.36,0.00}{\textbf{#1}}}
\newcommand{\OtherTok}[1]{\textcolor[rgb]{0.56,0.35,0.01}{#1}}
\newcommand{\PreprocessorTok}[1]{\textcolor[rgb]{0.56,0.35,0.01}{\textit{#1}}}
\newcommand{\RegionMarkerTok}[1]{#1}
\newcommand{\SpecialCharTok}[1]{\textcolor[rgb]{0.00,0.00,0.00}{#1}}
\newcommand{\SpecialStringTok}[1]{\textcolor[rgb]{0.31,0.60,0.02}{#1}}
\newcommand{\StringTok}[1]{\textcolor[rgb]{0.31,0.60,0.02}{#1}}
\newcommand{\VariableTok}[1]{\textcolor[rgb]{0.00,0.00,0.00}{#1}}
\newcommand{\VerbatimStringTok}[1]{\textcolor[rgb]{0.31,0.60,0.02}{#1}}
\newcommand{\WarningTok}[1]{\textcolor[rgb]{0.56,0.35,0.01}{\textbf{\textit{#1}}}}
\usepackage{longtable,booktabs,array}
\usepackage{calc} % for calculating minipage widths
% Correct order of tables after \paragraph or \subparagraph
\usepackage{etoolbox}
\makeatletter
\patchcmd\longtable{\par}{\if@noskipsec\mbox{}\fi\par}{}{}
\makeatother
% Allow footnotes in longtable head/foot
\IfFileExists{footnotehyper.sty}{\usepackage{footnotehyper}}{\usepackage{footnote}}
\makesavenoteenv{longtable}
\usepackage{graphicx}
\makeatletter
\def\maxwidth{\ifdim\Gin@nat@width>\linewidth\linewidth\else\Gin@nat@width\fi}
\def\maxheight{\ifdim\Gin@nat@height>\textheight\textheight\else\Gin@nat@height\fi}
\makeatother
% Scale images if necessary, so that they will not overflow the page
% margins by default, and it is still possible to overwrite the defaults
% using explicit options in \includegraphics[width, height, ...]{}
\setkeys{Gin}{width=\maxwidth,height=\maxheight,keepaspectratio}
% Set default figure placement to htbp
\makeatletter
\def\fps@figure{htbp}
\makeatother
\setlength{\emergencystretch}{3em} % prevent overfull lines
\providecommand{\tightlist}{%
  \setlength{\itemsep}{0pt}\setlength{\parskip}{0pt}}
\setcounter{secnumdepth}{5}
\usepackage{booktabs}
\ifLuaTeX
  \usepackage{selnolig}  % disable illegal ligatures
\fi
\usepackage[]{natbib}
\bibliographystyle{plainnat}

\usepackage{amsthm}
\newtheorem{theorem}{Theorem}[chapter]
\newtheorem{lemma}{Lemma}[chapter]
\newtheorem{corollary}{Corollary}[chapter]
\newtheorem{proposition}{Proposition}[chapter]
\newtheorem{conjecture}{Conjecture}[chapter]
\theoremstyle{definition}
\newtheorem{definition}{Definition}[chapter]
\theoremstyle{definition}
\newtheorem{example}{Example}[chapter]
\theoremstyle{definition}
\newtheorem{exercise}{Exercise}[chapter]
\theoremstyle{definition}
\newtheorem{hypothesis}{Hypothesis}[chapter]
\theoremstyle{remark}
\newtheorem*{remark}{Remark}
\newtheorem*{solution}{Solution}
\begin{document}
\maketitle

{
\setcounter{tocdepth}{1}
\tableofcontents
}
\hypertarget{presentaciuxf3n}{%
\chapter{Presentación}\label{presentaciuxf3n}}

El Área de Incidencia e Investigación de la EGPP- PUCP presenta ``Mitos y verdades de las Elecciones Subnacionales en el Perú'', la cual tiene como objetivo presentar a detalle los principales mitos sobre los últimos seis procesos electorales subnacionales.

\begin{itemize}
\item
  Elecciones Regionales y Municipales 2002
\item
  Elecciones Regionales y Municipales 2006
\item
  Elecciones Regionales y Municipales 2010
\item
  Elecciones Regionales y Municipales 2014
\item
  Elecciones Regionales y Municipales 2018
\item
  Elecciones Regionales y Municipales 2022
\end{itemize}

\hypertarget{mito-1}{%
\chapter{Mito 1}\label{mito-1}}

\textbf{El porcentaje de participación electoral ha disminuído.}

Se observa que la participación electoral ha disminuído a partir de las Elecciones Regionales y Municipales del 2010. En el 2022, se registró el porcentaje más bajo de participación electoral.

\includegraphics[width=19.15in]{mito1_1}

\includegraphics{bookdownproj_files/figure-latex/unnamed-chunk-2-1.pdf}

\includegraphics{bookdownproj_files/figure-latex/unnamed-chunk-6-1.pdf}

\includegraphics{bookdownproj_files/figure-latex/unnamed-chunk-7-1.pdf}

\hypertarget{mito2}{%
\chapter{Mito 2}\label{mito2}}

\textbf{El porcentaje del voto no válido supera al voto de la organización política ganadora.}

El voto nulo o blanco están ganado en porcentaje al ganador
¿Representatividad?
Porcentaje votos no válidos (de nulos y blancos) (el total es en base a votos de emitidos)

En el 2002, en 5 regionales , el voto nulo ganó al voto del ganador.

\hypertarget{elecciones-regionales}{%
\section{Elecciones Regionales}\label{elecciones-regionales}}

\includegraphics{bookdownproj_files/figure-latex/unnamed-chunk-10-1.pdf}

\hypertarget{porcentaje-de-votos-no-vuxe1lidos-por-cada-regiuxf3n}{%
\subsection{Porcentaje de votos no válidos por cada región}\label{porcentaje-de-votos-no-vuxe1lidos-por-cada-regiuxf3n}}

\includegraphics{bookdownproj_files/figure-latex/unnamed-chunk-14-1.pdf}

\includegraphics{bookdownproj_files/figure-latex/unnamed-chunk-15-1.pdf}

\hypertarget{porcentaje-de-votos-no-vuxe1lidos-versus-porcentaje-de-votos-por-la-organizaciuxf3n-poluxedtica-ganadora-en-las-elecciones-regionales}{%
\subsection{Porcentaje de votos no válidos versus Porcentaje de votos por la organización política ganadora en las Elecciones Regionales}\label{porcentaje-de-votos-no-vuxe1lidos-versus-porcentaje-de-votos-por-la-organizaciuxf3n-poluxedtica-ganadora-en-las-elecciones-regionales}}

\includegraphics{bookdownproj_files/figure-latex/unnamed-chunk-18-1.pdf}

\hypertarget{elecciones-provinciales}{%
\section{Elecciones Provinciales}\label{elecciones-provinciales}}

\includegraphics{bookdownproj_files/figure-latex/unnamed-chunk-22-1.pdf}

\hypertarget{porcentaje-de-votos-no-vuxe1lidos-versus-porcentaje-de-votos-por-la-organizaciuxf3n-poluxedtica-ganadora-en-las-elecciones-provinciales}{%
\subsection{Porcentaje de votos no válidos versus Porcentaje de votos por la organización política ganadora en las Elecciones Provinciales}\label{porcentaje-de-votos-no-vuxe1lidos-versus-porcentaje-de-votos-por-la-organizaciuxf3n-poluxedtica-ganadora-en-las-elecciones-provinciales}}

\includegraphics{bookdownproj_files/figure-latex/unnamed-chunk-26-1.pdf}

\hypertarget{elecciones-distritales}{%
\section{Elecciones Distritales}\label{elecciones-distritales}}

\includegraphics{bookdownproj_files/figure-latex/unnamed-chunk-29-1.pdf}

\hypertarget{porcentaje-de-votos-no-vuxe1lidos-versus-porcentaje-de-votos-por-la-organizaciuxf3n-poluxedtica-ganadora-en-las-elecciones-distritales}{%
\subsection{Porcentaje de votos no válidos versus Porcentaje de votos por la organización política ganadora en las Elecciones Distritales}\label{porcentaje-de-votos-no-vuxe1lidos-versus-porcentaje-de-votos-por-la-organizaciuxf3n-poluxedtica-ganadora-en-las-elecciones-distritales}}

NOTA: EL CODIGO ESTÁ TENIENDO PROBLEMAS

\includegraphics{bookdownproj_files/figure-latex/unnamed-chunk-31-1.pdf}

\hypertarget{mito-3}{%
\chapter{Mito 3}\label{mito-3}}

\textbf{Los movimientos regionales predominan en la oferta de candidaturas a nivel subnacional}

\hypertarget{porcentaje-de-participaciuxf3n-de-partidos-poluxedticos-por-eleccion-y-procedencia}{%
\section{Porcentaje de participación de partidos políticos por Eleccion y procedencia}\label{porcentaje-de-participaciuxf3n-de-partidos-poluxedticos-por-eleccion-y-procedencia}}

\includegraphics{bookdownproj_files/figure-latex/unnamed-chunk-34-1.pdf}

\hypertarget{mito-4}{%
\chapter{Mito 4}\label{mito-4}}

\textbf{Los movimientos regionales predominan en la candidaturas victoriosas a nivel subnacional}

\hypertarget{mito-5}{%
\chapter{Mito 5}\label{mito-5}}

\textbf{La competencia electoral es alta.}

Here is an equation.

\begin{equation} 
  f\left(k\right) = \binom{n}{k} p^k\left(1-p\right)^{n-k}
  \label{eq:binom}
\end{equation}

You may refer to using \texttt{\textbackslash{}@ref(eq:binom)}, like see Equation \eqref{eq:binom}.

\hypertarget{theorems-and-proofs}{%
\section{Theorems and proofs}\label{theorems-and-proofs}}

Labeled theorems can be referenced in text using \texttt{\textbackslash{}@ref(thm:tri)}, for example, check out this smart theorem \ref{thm:tri}.

\begin{theorem}
\protect\hypertarget{thm:tri}{}\label{thm:tri}For a right triangle, if \(c\) denotes the \emph{length} of the hypotenuse
and \(a\) and \(b\) denote the lengths of the \textbf{other} two sides, we have
\[a^2 + b^2 = c^2\]
\end{theorem}

Read more here \url{https://bookdown.org/yihui/bookdown/markdown-extensions-by-bookdown.html}.

\hypertarget{callout-blocks}{%
\section{Callout blocks}\label{callout-blocks}}

The R Markdown Cookbook provides more help on how to use custom blocks to design your own callouts: \url{https://bookdown.org/yihui/rmarkdown-cookbook/custom-blocks.html}

\hypertarget{mito-6}{%
\chapter{Mito 6}\label{mito-6}}

\textbf{La fragmentación política es alta.}

\hypertarget{publishing}{%
\section{Publishing}\label{publishing}}

HTML books can be published online, see: \url{https://bookdown.org/yihui/bookdown/publishing.html}

\hypertarget{pages}{%
\section{404 pages}\label{pages}}

By default, users will be directed to a 404 page if they try to access a webpage that cannot be found. If you'd like to customize your 404 page instead of using the default, you may add either a \texttt{\_404.Rmd} or \texttt{\_404.md} file to your project root and use code and/or Markdown syntax.

\hypertarget{metadata-for-sharing}{%
\section{Metadata for sharing}\label{metadata-for-sharing}}

Bookdown HTML books will provide HTML metadata for social sharing on platforms like Twitter, Facebook, and LinkedIn, using information you provide in the \texttt{index.Rmd} YAML. To setup, set the \texttt{url} for your book and the path to your \texttt{cover-image} file. Your book's \texttt{title} and \texttt{description} are also used.

This \texttt{gitbook} uses the same social sharing data across all chapters in your book- all links shared will look the same.

Specify your book's source repository on GitHub using the \texttt{edit} key under the configuration options in the \texttt{\_output.yml} file, which allows users to suggest an edit by linking to a chapter's source file.

Read more about the features of this output format here:

\url{https://pkgs.rstudio.com/bookdown/reference/gitbook.html}

Or use:

\begin{Shaded}
\begin{Highlighting}[]
\NormalTok{?bookdown}\SpecialCharTok{::}\NormalTok{gitbook}
\end{Highlighting}
\end{Shaded}


  \bibliography{book.bib,packages.bib}

\end{document}
